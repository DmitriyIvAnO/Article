\documentclass[12pt,a4paper]{article}
\begin{document}
\begin{center}
{\LARGE Comb filter}
\end{center}
In signal processing, a comb filter adds a delayed version of a signal to itself, causing constructive and destructive interference. The frequency response of a comb filter consists of a series of regularly spaced notches, giving the appearance of a comb.
\section{\Large Applications}
Comb filters are used in a variety of signal processing applications. These include:
\begin{itemize}
\item Cascaded integrator–comb (CIC) filters, commonly used for anti-aliasing during interpolation and decimation operations that change the sample rate of a discrete-time system.
\item 2D and 3D comb filters implemented in hardware (and occasionally software) for PAL and NTSC television decoders. The filters work to reduce artifacts such as dot crawl.
\item Audio effects, including echo, flanging, and digital waveguide synthesis. For instance, if the delay is set to a few milliseconds, a comb filter can be used to model the effect of acoustic standing waves in a cylindrical cavity or in a vibrating string.
\item In astronomy the astro-comb promises to increase the precision of existing spectrographs by nearly a hundredfold.
\end{itemize}
In acoustics, comb filtering can arise in some unwanted ways. For instance, when two loudspeakers are playing the same signal at different distances from the listener, there is a comb filtering effect on the signal.[1] In any enclosed space, listeners hear a mixture of direct sound and reflected sound. Because the reflected sound takes a longer path, it constitutes a delayed version of the direct sound and a comb filter is created where the two combine at the listener.
\section{\Large Technical discussion}
Comb filters exist in two different forms, feedforward and feedback; the names refer to the direction in which signals are delayed before they are added to the input.\\Comb filters may be implemented in discrete time or continuous time; this article will focus on discrete-time implementations; the properties of the continuous-time comb filter are very similar.
\end{document}